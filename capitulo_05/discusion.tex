\fancyhead{}
\fancyfoot{}
\cfoot{\thepage}

\lhead{Discusión}

\chapter{Discusión}

En este capítulo, que también suele denominarse ``Conclusiones'', se derivan las conclusiones, se explicitan recomendaciones para otros estudios (por ejemplo, sugerir nuevas preguntas, muestras, instrumentos, líneas de investigación, etc.) y se indica lo que sigue y lo que debe hacerse. Se analiza la posibilidad de extender los resultados a una población mayor que la del estudio. Se evalúan las implicaciones, se establece la manera como se respondieron las preguntas de investigación, si se cumplieron o no los objetivos, se relacionan los resultados con los estudios existentes (vincular con el marco teórico y señalar si los resultados coinciden o no con la literatura previa, en qué sí y en qué no). Se reconocen las limitaciones de la investigación, se destaca la importancia y significado de todo el estudio y la forma como encaja en el conocimiento disponible. Se explican los resultados inesperados y cuando no se verificaron las hipótesis es necesario señalar o al menos especular sobre las razones. Recordar que no se deben repetir aquí los resultados sino que se los debe interpretar. La discusión debe redactarse de tal manera que se facilite la toma de decisiones respecto de una teoría, un curso de acción o una problemática. Resumiendo, este capítulo puede ser conceptualmente y dividido en al menos tres secciones, como se ilustra a continuación.

\section{Logros alcanzados}
Descripción de los principales descubrimientos obtenidos como producto de la interpretación de los resultados de la investigación.
\section{Solución del problema de investigación}
Aquí se realiza la discusión propiamente dicha, respondiendo al problema planteado e indicando el nivel de satisfacción de la solución lograda.
\section{Sugerencias para futuras investigaciones}
Todo trabajo de investigación, genera invariablemente como producto colateral, otras interrogantes que suelen ameritar seguir con la investigación. Esto es derivado del caracter abierto, \textit{i.e.}, inacabado, del conocimiento científico. En esta sección se acostumbra hacer referencia a posibles seguimientos de la investigación indicando las interrogantes que conforman nuevos problemas pasibles de ser indagados.   