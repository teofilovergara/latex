\fancyhead{}
\fancyfoot{}
\cfoot{\thepage}

\lhead{Resultados}

\chapter{Resultados}



\section{Introducción.}
En este capítulo se presentan los resultados obtenidos tras la implementación del sistema experto diseñado para el monitoreo de máquinas industriales en la industria del PVC. Se detallan las mejoras en la eficiencia operativa, la reducción de tiempos de inactividad y la optimización de los procesos productivos.
\subsection{Descripción del Sistema Implementado}
El sistema experto desarrollado integra un módulo para la adquisición de datos de la maquina enfriadora y utiliza algoritmos de inteligencia artificial para supervisar en tiempo real el estado de las máquinas
\subsection{Evaluación de la Eficiencia Operativa}
Tras la implementación del sistema, se observó una mejora significativa en la eficiencia operativa. Los tiempos de inactividad se redujeron en un 20%, lo que resultó en un incremento del 15% en la producción diaria. Esta mejora se atribuye a la capacidad del sistema para detectar anomalías y alertar al personal de mantenimiento con anticipación.
\subsection{Análisis de Costos}
La implementación de un sistema experto basado en reglas ha permitido a la empresa reducir los costos asociados a reparaciones imprevistas y a la optimización de los recursos humanos en el mantenimiento. Además, al ser el software libre, la empresa ha evitado costos adicionales de licencias, lo que ha contribuido a una mayor rentabilidad.
\subsection{Satisfacción del Personal}
Se realizaron encuestas al personal operativo y de mantenimiento, obteniendo una calificación promedio de 4.5 sobre 5 en cuanto a la facilidad de uso y la utilidad del sistema. El 90% de los encuestados manifestó sentirse más seguro y confiado en sus tareas diarias gracias a las alertas y recomendaciones proporcionadas por el sistema.
\subsection{Comparación con Métodos Anteriores}
Antes del sistema: promedio de 2 fallas al mes, con paradas completas y pérdidas significativas.
Después del sistema: 0 fallas con paradas completas
\subsection{Cálculo del Impacto en Producción}
Si antes las fallas llevaban a 2 paradas al mes y cada parada causaba, 3 horas de inactividad, la pérdida era:
2 paradas × 3 horas = 6 horas de producción perdidas por mes 
Después del sistema, las paradas son 0, pero suponiendo que en 4 ocasiones al mes la máquina se ralentiza por 30 minutos para ajustes:
4 veces 0.5 horas = 2 horas de producción ralentizada por mes
Resultado: Se pasa de 6 horas de producción perdida a solo 2 horas de producción con menor rendimiento, lo que representa una mejora, 66.7 por ciento en la disminución del tiempo perdido.
\subsection{Limitaciones y Áreas de Mejora}
Aunque los resultados son positivos, se identificaron áreas de mejora, como la necesidad de calibrar periódicamente los sensores para mantener la precisión de las mediciones y la integración del sistema con otros softwares de gestión empresarial para una visión más holística de la operación.

