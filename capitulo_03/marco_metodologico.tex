\fancyhead{}
\fancyfoot{}
\cfoot{\thepage}


\lhead{Método}

\chapter{Método}

Este capítulo describe cómo fue realizado el trabajo de investigación, e incluye genéricamente los tópicos descriptos brevemente en las siguientes secciones.

\section{Enfoque}
La naturaleza del problema define el enfoque y el tipo de la investigación; estos pueden ser \cite{sampieri}:
\begin{itemize}
\item Cuantitativa; usa la recolección de datos para probar hipótesis, con base en la medición numérica y el análisis estadístico, para establecer patrones de regularidad y probar teorías. 
\item Cualitativa; utiliza la recolección de datos sin medición numérica para descubrir o afinar preguntas de investigación en el proceso de investigación.
\item mixta o cualicuantitativa; en general, toda investigación posee aspectos cualitativos (siempre) y cuantitativos, de manera que en la medida en que este compartimiento de aspectos se encuentre balanceado, se puede hablar de un tipo mixto cualicuantitativo.
En la FPUNE, dada la naturaleza tecnológica de los estudios en esta casa de estudios superiores, generalmente los trabajos de investigación adoptan un enfoque cuantitativo; a veces se adopta el enfoque mixto o el cualitativo por ser más adecuado a la naturaleza del problema estudiado.
\end{itemize}

\section{Alcance de la investigación cuantitativa}
El alcance de la investigación consiste en una medida de causalidad de la misma, entendida la causalidad como la relación causa efecto existente entre las variables, siendo el alcance; el grado de identificación de esta relación. La medida de causalidad puede variar dentro de límites de un continuo con varios grados caracterizados, estos grados de alcance bien caracterizados son: exploratorio, descriptivo, correlacional y explicativo. Las investigaciones exploratorias sirven para preparar el terreno y por lo común anteceden a investigaciones con alcances más profundos. Las investigaciones descriptivas pueden ser base de investigaciones correlacionales, si no explicativos; y así también las investigaciones correlacionales pueden proporcionar información para llevar a cabo investigaciones explicativas. Las investigaciones explicativas explicitan relaciones causa efecto, generan un sentido de entendimiento y son altamente estructurados. Es posible que una investigación se inicie como exploratoria, después puede ser descriptiva, luego correlacional y terminar siendo explicativa. El alcance depende fundamentalmente de dos factores: el estado del conocimiento sobre el problema de investigación, mostrado por la revisión de la literatura, así como la perspectiva que se pretenda dar a la investigación.

\subsection{Alcance exploratorio.}
La medida de este alcance abarca la exploración de problemas generalmente poco conocidos, a veces difíciles de conocer.

\subsection{Alcance descriptivo.}
La medida de este alcance abarca la descripción del fenómeno, situación, contexto o evento; detalla cómo es y cómo se manifiesta. Busca especificar propiedades, características y rasgos importantes. Describe tendencias de un grupo o población. Es útil para mostrar con precisión los ángulos o dimensiones de un fenómeno, suceso, comunidad, contexto o situación.

\subsection{Alcance correlacional.}
La profundidad de este alcance busca establecer relaciones entre variables sin precisar sentido de causalidad, es decir, no analiza relación causal.

Un ejemplo de este alcance es una investigación que busca averiguar cómo se relacionan las calificaciones de los alumnos de un grado, en las asignaturas: Castellano y Matemática.

\subsection{Alcance explicativo.}
La profundidad de este alcance busca establecer relaciones entre variables precisando sentido de causalidad, es decir, analiza relación entre causa y efecto entre variables.

Un ejemplo de este alcance es una investigación que busca averiguar la relación entre urbanización y alfabetismo en un país, para ver qué variables macrosociales definen el grado de alfabetización de la población del país.

\section{Diseño}
Es el plan o estrategia que se desarrolla para obtener la información que se requiere en una investigación, generalmente para verificar la hipótesis. La precisión, amplitud y profundidad de la información obtenida varía en función del diseño elegido \cite{sampieri}.

En la literatura sobre investigación cuantitativa es posible encontrar diferentes clasificaciones de los diseños; los autores \cite{sampieri} adoptan la siguiente clasificaciòn: investigación experimental e investigación no experimental. A su vez, la primera puede dividirse de acuerdo con las clásicas categorías de Campbell y Stanley (1966) en: preexperimentos, experimentos ``puros'' y cuasiexperimentos. La investigación no experimental, siempre de acuerdo con \cite{sampieri}, se subdivide en diseños transversales y diseños longitudinales.

\vspace{.5 cm}

\textbf{Ejemplo de diseño en una investigación tecnológica formativa.}

Aún más que en la investigación en ciencias básicas, es en la investigación tecnológica donde se puede apreciar la importancia del diseño para obtener un buen producto o servicio. Cabe entonces ilustrarlo con un ejemplo tomado dentro de esta última forma de investigación desde la referencia \cite{lan}.

\vspace{.5 cm}

\newacronym{lan}{LAN}{Local Area Network}
\glsreset{lan} % reinicia el banderín del primer uso

\textbf{\emph{Metodología para implementar red de área local. (\gls{lan}\@)}}\footnote{Por brevedad, solo se desarrolla la etapa de diseño.}

\vspace{.3 cm}

Hoy en día, como nunca antes, el ser social necesita estar informado. Para estudiar problemas y tomas de decisiones es necesario disponer de datos precisos, en el lugar y en el instante preciso. En gran medida se logra lo anterior con las redes de computadoras, cuyo objetivo fundamental es compartir recursos e información pues ofrecen acceso a servicios universales de datos tales como: bases de datos, correo electrónico, transmisión de archivos y boletines electrónicos; eliminando el desplazamiento de los individuos en la búsqueda de información y aumentando la capacidad de almacenamiento disponible por cada usuario en un momento determinado.

Un gran porcentaje de las redes de computadoras se usan para la transmisión de información científica siendo una vía rápida y económica de divulgar resultados y de discutir con otros especialistas afines sobre un tema en cuestión. En este trabajo en particular se aborda la metodología a seguir para la implementación de redes de computadoras de área local; las cuales cumplen todos los objetivos planteados a una escala reducida ya que son propiedad de una sola organización (un solo centro administrativo o fabril) abarcando zonas geográficas de algunos kilómetros como máximo. La experiencia en el campo de \glspl{lan} en el ámbito universitario, donde las mismas se emplean para la gestión administrativa y económica, para la transmisión de información científica y para la enseñanza; ha dejado claro que el diseño, la instalación y puesta a punto de una \gls{lan} suele ser un proceso cuidadoso del cual depende en grado sumo que se cumplan los objetivos para los que se invirtió en dicha red.

Para su comprensión el trabajo se divide en cinco partes o etapas:
\begin{itemize}
\item Etapa de estudio,
\item Etapa de diseño.\footnote{Solo se desarrolla esta etapa.}
\item Etapa de elaboración de la solicitud de oferta y selección del vendedor,
\item Etapa de instalación y puesta en funcionamiento,
\item Etapa de análisis de las prestaciones y evaluación de los resultados.
\end{itemize}
 
Una vez concluida la primera etapa y aprobado el presupuesto de la red es necesario realizar la etapa del \textit{diseño} de la \gls{lan} para lo cual se deben seguir los siguientes pasos:

\renewcommand{\labelitemi}{$-$}

\begin{itemize}
\item Seleccionar la(s) topología(s) y norma(s) de red a emplear,
\item Seleccionar el soporte de transmisión a utilizar,
% \item Seleccionar el \acrlong{sored} que se usará,
\item Analizar la necesidad de emplear técnicas de conectividad,
\item Considerar ampliaciones futuras de la red,
\item Realizar una evaluación primaria del tráfico,
\item Contemplar las necesidades del personal involucrado en la red,
\item Modificar, de ser necesario, el flujo de la información y seleccionar el software de aplicación.
\end{itemize}

\textit{Seleccionar la topología.} Este paso, el cual es dependiente de los resultados del anterior. Las tres topologías más empleadas son: bus, estrella y anillo; mientras que las normas más comunes son: Ethernet, Token Ring y ArcNet. La selección de los aspectos anteriores trae aparejado escoger la velocidad de transmisión, la distancia máxima a emplear, el método de control de acceso al medio, etc. La elección se realiza a partir de la necesidad particular y de un amplio conocimiento de las topologías y normas existentes. 

\textit{Seleccionar el Soporte de Transmisión.} Esto está muy relacionado con la norma a emplear y con las características de los puntos a conectar. Es vital realizar una selección adecuada pues una opción equivocada comprometería la eficacia y la velocidad de la transferencia de datos. Para la elección de uno u otro medio de transmisión se debe tomar entre otras cosas las dimensiones de la instalación, el costo, la evolución tecnológica estimada, la facilidad de instalación y el grado de hostilidad electromagnética presente en el entorno. 

\newacronym{sored}{SOR}{Sistema Operativo de Red}
\glsreset{sored} % reinicia el banderín del primer uso

\textit{Seleccionar el \gls{sored}.}

Aunque el \gls{sored}  (del inglés NOS: Netware Operating System) NetWare predomina en el mundo, éste no es siempre la elección adecuada, debido a sus costos y características. En el mercado existen otros \glspl{sored} tales como: LAN Manager, LANServer, LANtastic, Vines, LINUX, Windows NT Server, Windows 2000 Server, etc.; los cuales poseen una determinada cuota de mercado. Para seleccionar el SOR adecuado se debe tener en cuenta:

\begin{itemize}
\item El nivel de confidencialidad que brinda a los datos,
\item Si es del tipo cliente-servidor o de igual a igual,
\item Grado de tolerancia a fallos que posee,
\item Memoria RAM necesaria en el servidor y en las estaciones de trabajo,
\item Facilidades de administración y diagnóstico que brinda,
\item Si posee o no sistema de correo electrónico,
\item Características de manipulación de colas de impresión.
\end{itemize}

\textit{Analizar la necesidad de emplear técnicas de conectividad.} Esto estará en función de las dimensiones de la organización, del tráfico a cursar y el tipo de equipamiento a interconectar entre otros aspectos. Es necesario conocer en profundidad dichas técnicas para realizar una adecuada selección entre repetidores, puentes, ruteadores, compuertas, servidores de acceso, etc. y lograr su correcta ubicación. La mejor solución muchas veces hace uso de más de un tipo de dispositivo de interconexión.

\textit{Considerar ampliaciones futuras de la red.} Aún cuando de forma inmediata no sea necesario extender la red ni conectarse a otros, ésta debe poseer la base para que a partir de ella, y en cualquier momento sea posible una ampliación o llegar a formar parte de otras redes.

\textit{Realizar una evaluación primaria del tráfico.} Aquí debe estimarse el tráfico que circulará en la red y analizar si el mismo no afecta el tiempo de acceso a la información ya otros recursos compartidos. Es importante que una vez instalada y puesta en funcionamiento la \gls{lan} se efectúen periódicamente estudios de este tipo.

\textit{Contemplar las necesidades del personal involucrado en la red.} Esto es muy importante pues en última instancia éste será el personal que utilizará la red y por lo tanto deben quedar satisfechas sus necesidades de forma tal que la nueva red sea un elemento que facilite su trabajo.

\textit{Modificar de ser necesario el flujo de información y seleccionar el software de aplicación.} Esto implica la modificación, como última opción, de la manera en que la información circula dentro de la organización y la definición del software de aplicación necesario, ya sea comercial o aquél que se encargará al personal especializado; que conozca las particularidades de la programación en ambiente multiusuario. El software encargado o adquirido debe ser de fácil instalación y aprendizaje. Además se debe velar porque sea posible tener acceso a posteriores actualizaciones y que éstas no sean caras.
