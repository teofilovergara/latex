\thispagestyle{empty}
\begin{center}
\begin{LARGE}
\textbf{Resumen}
\end{LARGE}
\end{center}
\begin{quotation}
Este trabajo presenta el desarrollo de un sistema experto basado en reglas para el monitoreo y control de máquinas críticas en entornos industriales. En particular, se enfoca en la supervisión de equipos utilizados en la producción de materiales plásticos, donde la precisión y la eficiencia operativa son esenciales para garantizar la calidad del producto final y minimizar costos asociados a fallos operativos.
El principal objetivo del proyecto es diseñar e implementar un sistema experto que permita detectar anomalías en tiempo real mediante la aplicación de reglas específicas adaptadas a las condiciones operativas del sistema. A partir del análisis de variables clave, como el estado de la bomba de agua, el compresor, el caudal y las temperaturas de operación, el sistema genera alertas precisas cuando se identifican patrones anómalos, facilitando la toma de decisiones preventivas.
Para el desarrollo del sistema, se empleó una base de datos que almacena reglas previamente definidas, permitiendo una evaluación dinámica de los estados de las máquinas. Se utilizaron metodologías de inferencia basadas en lógica de producción, lo que garantiza la flexibilidad del sistema ante diferentes escenarios operativos. Los resultados obtenidos demuestran que el sistema experto mejora significativamente la detección temprana de fallos, reduciendo tiempos de inactividad y costos de mantenimiento.
Como conclusión, la implementación de este sistema experto permite a las empresas industriales optimizar la gestión de sus recursos, mejorar la confiabilidad de sus equipos y adoptar un enfoque proactivo en el mantenimiento preventivo. Su integración con sistemas de monitoreo existentes refuerza la capacidad de respuesta ante posibles fallos, contribuyendo a una mayor productividad y reducción de riesgos operativos.
    
\vspace*{0.5cm}

\noindent {\bf Descriptores:} 1. Sistema experto, 2. Monitoreo industrial, 3. Reglas de inferencia, 4. Optimizacion operativa

\end{quotation}

