\fancyhead{}
\fancyfoot{}
\lhead{Introducción}
\cfoot{\thepage}

\chapter{Introducción}

Este capítulo típicamente realiza la presentación de todo el Trabajo Final de Grado (TFG), excepto por las conclusiones que no deben ser adelantadas aquí. Se considera este capítulo como el inicio de la parte textual del informe del trabajo, toda la redacción preliminar a la introducción corresponde así a la parte pretextual del mismo. Debería incluir, generalmente en este orden \cite{sampieri}.

\begin{table}[]
\begin{tabular}{lllll}
\textbf{aaa} & \textbf{bbbb} & \textbf{ccc} &  &  \\
11           & 22            & 33           &  &  \\
             &               &              &  &  \\
             &               &              &  & 
\end{tabular}
\end{table}
\section{Motivación}
La motivación que condujo al autor a seleccionar el tema y emprender la investigación. Así, se trata de un contexto dependiente enteramente de los gustos e intereses propios del autor.

La motivación que condujo al autor a seleccionar el tema y emprender la investigación. Así, se trata de un contexto dependiente enteramente de los gustos e intereses propios del autor.

   \section{Definición del problema}
Se trata de la presentación del tema y breve definición del problema, dejando para más adelante el desarrollo detallado de toda la problemática. Se acostumbra formular interrogantes o preguntas de investigación que señalan la orientación del tratamiento.

\section{Objetivos, hipótesis, justificación y delimitación del alcance del tratado.}
Es importante una clara definición de cada uno de estos tópicos para facilitar la comprensión de toda la obra. Esto otorga una visión global del trabajo e indica qué de resultados son buscados con el desarrollo del trabajo

\section{Descripción de los contenidos por capítulo.} 
Usualmente, el capítulo termina anunciando brevemente el contenido de los restantes capítulos.

